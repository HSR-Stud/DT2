\section{Parametrisierung}
VHDL kann Parameter an ein Modell �bergeben. Deklariert in
Schnittstellenbeschreibung. Dies macht den Code Flexibler und erlaubt die
wieder Verwendung von von Code\\

\subsection{Anwendung}
\begin{multicols}{2}
	\lstinputlisting[language=VHDL,tabsize=2]{code/parametrisierung.vhdl}
\end{multicols}

\subsection{Unterschiede der Konzepte}
\begin{tabular}{p{3cm}p{7.5cm}p{7.5cm}}
	Parameter fixiert zu \textbf{Desingetime} & VHDL Design mit Konstanten
	&Counter $0 \ldots 4$\\
	Parameter fixiert zu \textbf{Compiletime} & Gernic Konzept �bermittelt
	Information zu compiletime & Wiederverwendbarer Z�hler mit Z�hlerbereich 0
	$\ldots$ maxcount \\
	Parameter fixiert zur \textbf{Runtime} & Signale �bermitteln Informationen zur
	Runtime & Z�hlerbereich kann �ber Schalter eingestellt werden.
\end{tabular}


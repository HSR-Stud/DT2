\section{Realisierungs Methoden}
\subsection{ROM}
Mit einem ROM lassen sich sich rein kombinatorische Schaltungen in Form einer Look up Table realisieren.
\begin{itemize}
	\setlength{\itemsep}{1pt}
  \setlength{\parskip}{0pt}
  \setlength{\parsep}{0pt}
  
	\item Eingangsvariablen = Adresse
	\item Speicherwert = Ausgang (programmierbar)
\end{itemize}



\subsection{PLD}
Programmierbares Device aus AND und OR-Matrix, mindestens eine Matrix programmierbar.
\begin{itemize}
	\setlength{\itemsep}{1pt}
  \setlength{\parskip}{0pt}
  \setlength{\parsep}{0pt}
  
	\item PAL $\rightarrow$ OR-Matrix fest, AND-Matrix programmierbar, Fuses
	\item PLA $\rightarrow$ OR und AND Matrix frei programmierbar, Fuses
	\item GAL $\rightarrow$ Wie PLA plus programmierbare Ausgangsnetzwerke (Tristate), EEPROM
\end{itemize}
SPLD (Simple PLD): F"ur Funktionen die als DNF vorliegen geeignet, heute gr"osstenteils von CPLD und FPGA verdr"angt.

\subsection{CPLD (Complex PLD)}
\begin{itemize}
	\setlength{\itemsep}{1pt}
  \setlength{\parskip}{0pt}
  \setlength{\parsep}{0pt}
  
	\item Verbund PLD Makrozellen die mit Bussen verbunden sind, Speicherung der Konfiguration in Flash.
	\item	Durch regelm"assige Struktur sind Signallaufzeiten vorhersagbar.
	\item Wegen grosser Zahl an Logikbl"ocken sehr gut f"ur parallele Prozesse geeignet.
\end{itemize}

\subsection{FPGA}
2D-Array von Logikbl"ocken, die "uber Routing Kanal und Schaltmatrizen miteinander und mit I/O verbunden werden.
\begin{itemize}
	\setlength{\itemsep}{1pt}
  \setlength{\parskip}{0pt}
  \setlength{\parsep}{0pt}
  
	\item Logikblock (LogicCell) $\rightarrow$ Lookuptable mit D-FlipFlop, kann beliebige Funktionen ausführen
	\item Schaltmatrizen $\rightarrow$ programmierbare Verbindungen
	\item Makrozellen $\rightarrow$ Feste Funktionen z.B. Memory, Clock Managment...
\end{itemize}
Die Konfiguration wird im RAM gespeichert (flüchtig). D.h. bei jedem Boot muss der Code von einem Festspeicher geladen werden.

%\subsection{Xilinx Spartan 3}
%\begin{itemize}
%	\setlength{\itemsep}{1pt}
%  \setlength{\parskip}{0pt}
%  \setlength{\parsep}{0pt}
%
%	\item Logic Cell (LC): Kleinste Einheit, enth"alt LUT mit 4 Eing"angen und eime D-FlipFlop. LUT kann als 16x1 bit SRAM oder Schieberegister 		konfiguriert werden. Zus"atzlich pro LC CarryLogic und MUX.
%	\item Slice: 1Slice = 2 Logic Cell
%	\item Configurable logic bloc (CLB): 1 CLB = 4 Slices = 8 Locic Cells. \\
%	Inerhalb dieser Einheit existieren spezifische Verbindungsstrukturen.
%\end{itemize}
%
%\includegraphics[width=0.8\textwidth]{pics/fpgastruct}


\subsection{Semicustom IC}
\begin{itemize}
	\setlength{\itemsep}{1pt}
  \setlength{\parskip}{0pt}
  \setlength{\parsep}{0pt}
  
	\item Mikrozellen aus p- und n-FETs werden durch Verdrahtung zu Gates 			
		$\rightarrow$Gate-Array/Sea of Gates
	\item Gates k"onnen durch Verdrahtungskan"ale verbunden werden.
	\item Standardfunktionen k"onnen mit IP (intelegent properity)-Cores implementiert werden.
	\item In Mixed-Signal Arrays sind zus"atzlich spezifische Analogbauteile enthalten
\end{itemize}	
	
\subsection{Fullcustom IC}
V"ollig kundenspezifische ICs, oft werden IP-Cores f"ur Standardfunktionen verwendet. Digitale und analoge Komponenten auf einem IC m"oglich. Voll auf Anwendung anpassbare Eigenschaften (Stromverbrauch, Gr"osse, Geschwindigkeit etc.).

\subsection{Vergeichstabelle}

\begin{center}
	\includegraphics[width=0.40\textwidth]{pics/devicecomparetables}
	
	\begin{tabular}{|l|c|c|c|c|c|c|}
		\hline
		Kriterien & Standard Bauteile & ROM & PLD & FPGA & Semicustom & Fullcustom \\
		\hline
		Machbarkeit & ++ & - - & - - & + & + & +++ \\
		\hline
		Zeit Realisiertung & + & ++ & ++ & ++ & - & - - \\
		\hline
		Iterationszeit & - & ++ & ++ & ++ & - & - - \\
		\hline
		NRE & ++ & + & + & + & - & - - -\\
		\hline
		St"uckpreis & - - & + & + & - & + & +++ \\
		\hline
	\end{tabular}
\end{center}
	
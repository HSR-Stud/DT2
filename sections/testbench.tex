\section{Testbench}
\subsection{Eigenschaften}
\begin{itemize}
  \setlength{\itemsep}{1pt}
  \setlength{\parskip}{0pt}
  \setlength{\parsep}{0pt}
  \item Zu pr�fende Komponente wird unver�ndert aus Bibliothek �bernommen.
  (Component).
  \item Testbench muss verschiedene Architekturen einfach einbinden k�nnen.
  \item Innerhalb der Architektur der Testbench d�rfen auch nicht
  synthetisierbare VHDL Konstrukte verwendet werden. (z.B.: for-Schleife)
  \item Simulationszeit = Reale Zeit $\neq$ Rechenzeit der Simulation.
  \item Normalerweise 3 Prozesse (clock, stimuli, response).
\end{itemize}
\newpage
\subsection{Codebeispiel}
  \lstinputlisting[language=VHDL,tabsize=2]{code/testbench.vhdl}
\newpage  
\subsection{Timing Simulation}
\begin{multicols}{2}

  \subsubsection{Delta Time}
    Wird Verz�gerungszeit einer Operation nicht explizit angegeben, so wird mit
    $ \Delta $ time gerechnet (meistens 0).
  \subsubsection{Inertial}
    Signalzuweisung mit after:\\
    \lstinputlisting[language=VHDL,tabsize=2]{code/inertial_delay.vhdl}
    Y �ndert Wert nur, wenn \"Anderung des Eingangs l�nger wirksam ist als
    delay ist. Spikes werden nicht ber�cksichtigt.
  \subsubsection{Transport}
    Ausgang wird in jedem Fall um delay gegen�ber Eingang verschoben.
    \lstinputlisting[language=VHDL,tabsize=2]{code/transport_delay.vhdl}
\end{multicols}